\documentclass[a4paper,11pt]{article}

\usepackage[utf8]{inputenc}    % Pour que LaTeX comprenne les accents.
\usepackage{times}             % Police de caractères
\usepackage[english]{babel}     % Traitement du texte adapté aux règles typographiques
                               % de la langue donnée en option (e.g., pour l'espacement
                               % après les ponctuations
\usepackage[T1]{fontenc}
\usepackage{amsmath, amsthm, amssymb} 
\usepackage{dsfont}  % pour les indicatrices 
\usepackage{graphicx}                               
\sloppy              % Ne pas faire déborder les lignes dans la marge


\newtheorem{lemma}{Lemma}
\newtheorem{cons}{Corollary}
\newtheorem{theo}{Theorem}

\theoremstyle{definition}
\newtheorem{definition}{Definition}
\newtheorem{process}{Process}

\theoremstyle{remark}
\newtheorem{remark}{Remark}

\title{Throwing needles on a coloured plane.}

\begin{document}
\maketitle

\begin{abstract}  In this paper, we are interested in $n$-colourings of plane 
that maximize the probability that when throwing a needle in the plane the two 
endpoints of the needle have different colours. Our major result establish a 
correlation between $n$-colouring of plane and finite unit-distance graphs. 
It gives us some bounds for the optimal probability.\end{abstract}

\section{Introduction}
This idea of throwing a needle on a coloured floor was somehow motivated by 
the announcement of cell phones chargers that are only surfaces where a cell 
phone can be only launched to charge. Indeed, imagine that the endpoints 
represent two electrodes on our cell phone, and that the colours represent 
different electric potentials on the conductive plane : we would try to make 
the colouring optimal so that the cell phone can be charged with a big enough 
probability. Of course, actual implementations of that technology are not so 
simple

The problem we are interested in here is the following : if $c$ is a 
\emph{valid} colouring the plane with $n$ colours, in a  way that will be 
defined further, define $p(c)$ as the probability that when throwing a needle 
on $c$, both endpoints have the same colour. What is the minimal $p(c)$ and 
for what colourings is it obtained ?

This can be seen as a probabilistic version of the well known 
\emph{Hadwiger-Nelson problem}. This problem asks for the chromatic number 
of the euclidian plane, where unit-distance points are connected. Some 
relations with our problem will be presented in paragraph~\ref{hn}. One can 
first remark that knowing that the answer to the Hadwiger-Nelson problem is at 
most $7$, with a $7$-colouring of the plane such that two unit distance points 
always have different colours, the maximum probability we are looking for is 
$1$ if $n \geq 7$.

In section~\ref{def} we define the process more precisely. In particular, we 
ought to make some assumption on the colouring for the probability to make 
sense. Our main result, establishing links between finite graphs and our 
probability, is presented in section~\ref{equiv}. We then show some 
consequences, and try to generalise this idea in higher dimensions in
section~\ref{appli}. Finally, we study the problem of launching a needle on a finite 
table.

\section{Definitions}
\label{def}
First, we call \emph{colouring} a function $c$ from the euclidian plane to the 
set of colours $[| 1 ; n |]$. For the probability to be well defined, we will 
first require that every set $U_i$ for $i \in [| 1 ; n |]$ is Borel-measurable 
- equivalently, that the function $c$ itself be Borel-measurable. Since it's 
impossible to ``throw a needle uniformly on the plane'', we also choose to make 
the following assumption : 
the colouring is \textit{periodic} - that is, there is a basis of two vectors, 
$u$ and $v$, such that translations along these vectors leave the colouring 
unchanged.

Thereafter, we will always assume that the colouring is measurable and 
periodic, and we will denote $\mathbf{P}$ the parallelogram defined by these 
vectors. Needle throwing processes can then be considered on this parallelogram only.

The notion of a \emph{unit distance graph} will also be useful :
\begin{definition}

A graph $(V,E) $ is said to be a \emph{unit distance graph}, if it has a 
representation in the plane, with all edges having length $1$.
\end{definition}

We now define formally the needle-throwing process :
\begin{process}
Consider the following random variables :
\begin{itemize}
  \item let $A_1$ (the lowest point of the needle) be a point chosen uniformly 
  in $\mathbf{P}$ ;
  \item let $\theta$ be an independent angle chosen uniformly in $[0;\pi[$ ;
  \item let $A_2$ (the highest point of the needle) be $A_1 + e^{i \theta}$.
\end{itemize}
\end{process}

Note that $A_2$ may be outside of $\mathbf{P}$. In that case, $c(A_2)$ is 
defined using the periodicity of the colouring.
Our goal is to evaluate and minimize the probability that both ends have the 
same color.

\begin{remark}
Note that there are several couples of vectors $(u,v)$ forming an appropriate 
basis for the colouring. Nonetheless, it's clear that probabilities do not 
depend on this choice. 
\end{remark}

We consider the following, second process of a unit distance graph on the 
plane, and then of a needle in the graph :
\begin{process}
Given a unit distance graph $(G,V)$, label all the edges with numbers from 
$1$ to $|E|$. Choose : 
\begin{itemize}
\item one point $A_0$ uniformly on the initial parallelogram $\mathbf{P}$ for 
the origin of the graph.
\item an angle $\theta$ uniformly in $[0;2\pi[$, to rotate the graph
\item and an index $i$ in $[|1;|E||]$.
\end{itemize}
Then, rotate the graph by the angle $\theta$, translate it so that the origin 
falls on  $A_0$, and take the needle corresponding to the position of $i^{th}$ 
edge of the graph obtained. 
\end{process}
\vspace{1 cm} 

\section{Process equivalence}
\label{equiv}
\begin{lemma}\label{huitre}
Let $(A,B)$ denote the needle obtained by the original process, and
$(A',B')$ the one obtained by the second process. Let $c_1$ and $c_2$ denote 
two colors, and $c$, the application giving the color of a point. Then, we 
have :\\
 $$\mathbb{P}(c(A) = c_1 , c(B) = c_2) = \mathbb{P}(c(A') = c_1, c(B') = c_2) $$ \\
 \\
 In particular, this means that the probability doesn't depend on the graph 
 chosen. For a graph with only two vertices, we exactly get the initial process.
\end{lemma}

\begin{proof}
Conventions : 
\begin{itemize}
\item we take a graph $G=(V,E)$, and $m = |E|$, the number of edges.

\item as usual, $\mathbf{P}$ is the parallelogram obtained by the periodicity 
of the colouring.

\item $\mathcal{A}(\mathbf{P})$ is the area of the parallelogram we are 
considering, and $\mathcal{P}(\mathbf{P})$ its perimeter.

\item $i$ is the index of one of the edges of $G$.

\item we denote $z_i$ the complex coordinates of one vertex of the $i^{th}$ 
edge.

\item $\theta_i$ denotes the angle made by the $i^{th}$ edge of the graph. The 
complex coordinates of the vertices of the $i^{th}$ edges before 
rotation/translation are then $z_i $ and $z_i + e^{i.theta_i}$.

\end{itemize}

We prove it by the following series of equalities :

\begin{eqnarray*}
& & \mathbb{P}(c(A') = c_1, c(B') = c_2) \\
  &=& \frac{1}{m}\sum_{i=1}^{m} \mathbb{P}(c(A') = c_1, c(B') = c_2 | i)  \\
  &=& \frac{1}{m}\sum_{i=1}^{m}  \frac{1}{2\pi}\int_{\theta =0} ^{2\pi} \frac{1}{\mathcal{A}(\mathbf{P})}\int\int_{A_0 \in \mathbf{P}} \mathbb{P}(c(A') = c_1, c(B') = c_2 | i, \theta, A_0) d\theta dx dy \\  
  &=& \frac{1}{m 2\pi \mathcal{A}(\mathbf {P})}\sum_{i=1}^{m}\int_{\theta =0} ^{2\pi} \int\int_{A_0 \in \mathbf{P}} \mathds{1}(c(A_0 + z_i e^{i\theta}) = c_1, c(A_0 +z_i e^{i\theta} + e^{i (\theta + \theta_i)}) = c_2) d\theta dx dy \\  
    &=& \frac{1}{m 2\pi \mathcal{A}(\mathbf{P})}\sum_{i=1}^{m} \int_{\theta =0} ^{2\pi} \int\int_{A_0 \in \mathbf{P}} \mathds{1}(c(A_0) = c_1, c(A_0 + e^{i (\theta + \theta_i)}) = c_2 ) d\theta dx dy \hspace{1 cm} (*)\\ 
    &=& \frac{1}{m 2\pi \mathcal{A}(\mathbf{P})}\sum_{i=1}^{m} \int_{\theta =0} ^{2\pi} \int\int_{A_0 \in \mathbf{P}} \mathds{1}(c(A_0) = c_1, c(A_0 + e^{i \theta}) = c_2 ) d\theta dx dy \\ 
    &=& \frac{1}{2\pi \mathcal{A}(\mathbf{P})}\int_{\theta =0} ^{2\pi} \int\int_{A_0 \in \mathbf{P}} \mathds{1}(c(A_0) = c_1, c(A_0 + e^{i \theta}) = c_2 ) d\theta dx dy \\ 
\end{eqnarray*}
Which gives exactly the same probability as the first process (except that the 
angle is taken in $[0;2\pi[$ here, instead of $[0;\pi[$, but it doesn't make 
any difference when looking only at the colors).

The first two equalities are obtained by the law of total probability.
The equality $(*)$ is justified by the periodicity of our colouring.
\end{proof}

\begin{cons} \label{ineg}
From the previous theorem it follows that :
$$ \max_{g \in G} c_g(n) \leq \min_{c \in C_n} p(c) $$
where \begin{itemize} 
  \item $G$ is the set of finite unit-distance graphs 
  \item $C_n$ is the set of periodic n-colourings.
  \item $c_g(n)$ is the minimal number of edges with both ends with de
same colours when colouring g with n colours over the number of edges.
(For example : $\frac 1 {11}$ for Moser Spindle.)
  \item $p(c)$ is the probability of throwing a needle with the same two
ends, on the colouring $c$.
\end{itemize}

\end{cons}

\begin{proof}
Consider a graph G, n colours, when choosing a needle
uniformly on this graph the probability that both ends have the same
colours is greater than $c_G(n)$. Applying the previous theorem with G, 
it follows that the probability for any colouring is greater than $c_G(n)$    
\end{proof}
We still don't know wether or not this is actually an equality. 

\section{Applications} \label{appli}
\subsection{2 colours}


With two colors, we consider the second process with a triangle as our graph. 
As there is always at least two vertices with the same color, it's clear that 
the probability that the endpoints have the same color is at least 
$\frac{1}{3}$. Thanks to lemma~\ref{huitre}, we have the same inequality for 
the first process.

Let's prove the following colouring~\ref{color} shows that this bound is 
optimal. The colouring is constructed with parallel strips of size 
$l = \frac {\sqrt3}{2}.$ with the upper sides opened and the lower sides closed.   
%IMAGE DU COLORIAGE + preuve optimalité
\begin{figure}[h]
\center
\includegraphics[scale=0.5]{path6509.png}
\caption{\label{couleur} The parallel stripes $2$-colouring}
\end{figure}

Indeed, clearly, any unit equilateral triangle cannot have its three vertices
with the same colors, so Lemma~\ref{huitre} shows that we get a probability of 
$\frac{1}{3}$ indeed.
%CLASSE DE SOLUTIONS?

\subsection{3 colours}
\begin{figure}[h]
\label{pasteque}
\center
\includegraphics[scale=0.4]{T.png}
\caption{\label{color} The Moser spindle}
\end{figure}


 With three colors, we use the Moser spindle of Figure~\ref{pasteque} - that 
 has at least one of its $11$ edges with both ends of the same colour.
 We similarly use lemma~\ref{huitre} to get that in the case of three colours,
 the probability that both endpoints have the same colour is at least 
 $\frac{1}{11}$ with the first needle throwing process. 

We don't know if the previous bound is optimal. We believe that Figure~\ref{trois} 
can give a rather good $3$ colouring of the plane, for some well chosen length 
of the needle. Rough simulations have shown that this gave us a $p$ of about 
$0.12$ for a needle of length about $1.6$, if the edge of a hexagon is $1$. 
Still, we did not compute the exact optimum in that case because this bound is 
still far away from $\frac{1}{11}$.

\begin{figure}[h]
\center
\includegraphics[scale=0.5]{trois.png}
\caption{\label{trois} A hexagonal $3$-colouring and its parallelogram of periodicity}
\end{figure}
  %Mettre les exemples de coloriages, avec les probas obtenues.

\subsection{Other extensions}
\label{dim}
We can generalize the two processes in higher dimensions, and use the same 
proof to establish Lemma~\ref{huitre} and Corollary~\ref{ineg} in any 
dimension $d$.

%graphe unitaire en dimension d
%lancer un graphe/choisir une aiguille : choisir un point et d-1 angles.
%on a la même equalité

Remark that in higher dimension, better bonds may be provided, since new graphs can 
become unit-length. For example, with $3$ colours in dimension $3$, the regular 
tetrahedron witnesses a better lower bound 
of $\frac 1 6 >\frac 1 {11}$.

Since a unit-length graph in dimension $d$ is also
unit-length in higher dimensions, the bounds on $p$ are always sharper in higher
dimensions.
% mettre d'autres exemples plus généraux

\subsection{Connections with the Hadwiger-Nelson problem} \label{hn}
All results shown thereafter will be derived using the axiom of choice. 
This is important to notice, since the answer to the Hadwiger-Nelson problem is 
suspected to depend on the set of axioms used. Indeed, we will use the 
following theorem, which cannot be established without some form of axiom of 
choice :
\begin{theo}[De Bruijn - Erdős]
 A graph $G$ can be coloured with $k$ colours iff all of its finite subgraphs 
 can be coloured with $k$ colours.
\end{theo}
In other words, the chromatic number of a graph is the maximum chromatic number 
of its finite subgraphs.

If there exist a finite graph that cannot be coloured with $k$ colours, it 
follows from our study that the probability $p(c)$ for $c$ a periodic borelian 
$k$-colouring is always greater than a certain constant. Taking the 
contrapositive of this statement and using the De Bruijn-Erdős theorem, we get  
the interesting proposition :
\begin{cons}
 If there exist a sequence $(c_m)_{m \in \mathbb{N}}$ of periodic borelian 
 $k$-colourings such that $\lim_{m \to \infty} p(c_m) = 0$, then the 
 Hadwiger-Nelson graph can be coloured using $k$ colours.
\end{cons}
Note that the final colouring of the plane we get is not necessarily periodic 
or even borelian, since its existence relies on a theorem that highly depends 
on the axiom of choice.

\section{Finite table} \label{fini}

The previous process can be modified to match a somewhat more practical view. 
In this section, we consider the process of needle throwing in a finite table. 
Formally, it is defined as follows :


\begin{process}
Denote by $\mathbf{P}$ a parallelogram (representing the table). 
\begin{itemize}
\item Choose a point uniformly in $\mathbf{P}$ for the lower point of the needle
\item Choose an angle theta uniformly in $[0 ; \pi[$ such that the second end 
of the needle does not fall out of the table
\end{itemize}
\end{process}

\begin{definition}
Let $K(\mathbf{P},r)$ the set obtained by taking $\mathbf{P}$ 
and removing the border, that is a $r$-length stripe.
\end{definition}

\begin{figure}[h]
\center
\includegraphics[scale=0.5]{tablefinie.png}
\caption{\label{tablefinie} The set $K(\mathbf{P},r)$ in green}
\end{figure}

Similarly to what was done in the previous section, we define a second process, 
consisting in throwing a triangle, and then choosing a edge on this triangle.
This will provide a bound on the probability for two colours.

\begin{process}
We now define an other needle choosing process, obtained by launching a graph :
\begin{itemize}
  \item let $\theta '$ (the orientation of our triangle) be chosen uniformly 
  in $[0;\pi / 3[$ ;
  \item let $A'$ (the lowest vertex of the triangle) be a point in $\mathbf{P}$ such 
  that $A' + e^{i\theta '}$ is also in $P$, chosen uniformly among the points 
  having this property ;
  \item let $B' = A' + e^{i \theta '}$ and $C' = A' + e^{i (\theta ' + \pi / 3 ) }$ ;
  \item let $(A'_1,A'_2)$ (lowest and highest endpoints of the needle) be a 
  couple of points chosen uniformly between $(A',B')$, $(A',C')$, $(B',C')$.
\end{itemize}
Finally, let $p'$ be the probability that $A'_1$ and $A'_2$ have the same 
colour.
\end{process}


The result is the following :
\begin{lemma}
Denote by $p$ the probability that both ends have the same colour when throwing 
the needle on a finite table, then :
 $$ | p - p'| \leq \frac{\mathcal{A}(K(\mathbf{P},1))}{\mathcal{A}(\mathbf{P})} $$

In particular, this bound is sharp when the dimensions of the table goes to 
infinity. It follows immediately from this that : $p \geq \frac13 + \frac{\mathcal{A}(K(\mathbf{P},1))}{\mathcal{A}(\mathbf{P})} $ \\
The proof can be adapted to circular tables, or more generally to any table 
whose shape is not too complicated. \\
%Say something about wether it can be adapted or not to moser ?
\end{lemma}

\begin{proof}

\begin{eqnarray*}
|p' - p| 
  &=& | \mathbb{P}(A'_1 , A'_2 \in U_0) - \mathbb{P}(A_1 , A_2 \in U_0) + \mathbb{P}(A'_1 , A'_2 \in U_1) - \mathbb{P}(A_1 , A_2 \in U_1) | \\
  &\leq& | \mathbb{P}(A'_1 , A'_2 \in U_0) - \mathbb{P}(A_1 , A_2 \in U_0) | + | \mathbb{P}(A'_1 , A'_2 \in U_1) - \mathbb{P}(A_1 , A_2 \in U_1) | 
\end{eqnarray*}

Let's focus on the first term. 
Decompose the first probability in three terms corresponding to the final 
choice of $(A',B')$, $(A',C')$ or $(B',C')$, and the second one in three terms 
according to the position of $\theta$ :
\begin{eqnarray*}
& & \left| \mathbb{P}(A'_1 , A'_2 \in U_0) - \mathbb{P}(A_1 , A_2 \in U_0) \right| \\
&\leq& \frac{1}{3} \left| \mathbb{P}(A' \in U_0 , B' \in U_0) - \mathbb{P}(A_1 \in U_0 , A_1 + e^{i \theta} \in U_0 \ |\  \theta < \frac{\pi}{3} ) \right| \\
&+& \frac{1}{3} \left| \mathbb{P}(A' \in U_0 , C' \in U_0) - \mathbb{P}(A_1 \in U_0 , A_1 + e^{i \theta} \in U_0 \ |\  \frac{\pi}{3} \leq \theta < \frac{2 \pi}{3} ) \right| \\
&+& \frac{1}{3} \left| \mathbb{P}(B' \in U_0 ,C' \in U_0) - \mathbb{P}(A_1 \in U_0 , A_1 + e^{i \theta} \in U_0 \ | \  \frac{2 \pi}{3} > \theta ) \right|
\end{eqnarray*}

Once again, let's focus on the first term. When the points $A$ and 
$A'$ fall in $K(\mathbf{P},1)$, then the two points are uniformly distributed in 
$K(\mathbf{P},1)$, and independent of $\theta$ and $\theta'$. Thus we have : 
$$ \left| \mathbb{P}(A'_1 , A'_2 \in U_0) - \mathbb{P}(A_1 , A_2 \in U_0) \right| \leq \mathbb P(A \not \in K(\mathbf{P},1)) \leq \frac{\mathcal{A}(K(\mathbf{P},1))}{\mathcal{A}(\mathbf{P})} $$


\end{proof}

\section{Final notes}
In the infinite case, it may seem a bit disappointing that we chose to study 
only periodic colourings. One can indeed hope for some similar results for a 
larger class of colourings. For instance, we can look at the probability $p_R$ 
of obtaining two same coloured endpoints when launching the needle on the ball 
$B(0,R)$, and then say that a colouring is \emph{acceptable} if $p_R$ converges 
as $R \rightarrow \infty$. If we denote $p$ this limit, section~\ref{fini} 
easily shows that we get the same main results for $p$, especially
consequence~\ref{ineg} when taking $C_n$ the set of acceptable colourings with $n$ colours. 
It's also clear that any periodic colouring is acceptable. Still, we chose not 
to use that notion, because $p$ is not a probability, and shouldn't be 
considered as such. It may though be the most natural way to give a sense to 
the (impossible) idea of ``launching a needle uniformly on the whole plane''.

%\iffalse
%\subsection*{First proof}
%
%\begin{lemma}
% Let $p$ be the probability that both $A_1$ and $A_2$ have same colour. Let $\mathcal{A}(P)$ (resp. $\mathbf{P}(P)$) be the area (resp. perimeter) of $P$. We have :
% \[p \geq \frac{1}{3} - \frac{\mathbf{P}(P)}{\mathcal{A}(P)} \]
%\end{lemma}
%
%\begin{proof}
%The main idea is that an equilateral triangle must have at least one edge having both endpoints of the same colour.
%It also gives essentially the same results to choose an equilateral triangle randomly, and then to choose on of its edges as our needle.
%
%
%\begin{process}
%We therefore define a second needle choosing process :
%\begin{itemize}
%  \item let $\theta '$ (the orientation of our triangle) be chosen uniformly in $[0;\pi / 3]$ ;
%  \item let $A'$ (the lowest vertex of the triangle) be a point in $P$ such that $A' + e^{i\theta '}$ is also in $P$, chosen uniformly among the points having this property ;
%  \item let $B' = A' + e^{i \theta '}$ and $C' = A' + e^{i (\theta ' + \pi / 3 ) }$ ;
%  \item let $(A'_1,A'_2)$ (lowest and highest endpoints of the needle) be a couple of points chosen uniformly between $(A',B')$, $(A',C')$, $(B',C')$.
%\end{itemize}
%and $p'$ as the probability that $A'_1$ and $A'_2$ have the same colour.
%\end{process}
%
%Clearly, because there are at least two vertex of a triangle with the same color, we have the inequality $$p' \geq \frac{1}{3} $$.
%
%To conclude the proof, it suffices to prove that $|p' - p| \leq \frac{\mathcal{P}(P)}{\mathcal{A}(P)}$.
%If we denote $U_0$ (resp. $U_1$) the set of points $M$ such that $c(M) = 0$ (resp. $1$), we have :
%\begin{eqnarray*}
%|p' - p| 
%  &=& | \mathbb{P}(A'_1 , A'_2 \in U_0) - \mathbb{P}(A_1 , A_2 \in U_0) + \mathbb{P}(A'_1 , A'_2 \in U_1) - \mathbb{P}(A_1 , A_2 \in U_1) | \\
%  &\leq& | \mathbb{P}(A'_1 , A'_2 \in U_0) - \mathbb{P}(A_1 , A_2 \in U_0) | + | \mathbb{P}(A'_1 , A'_2 \in U_1) - \mathbb{P}(A_1 , A_2 \in U_1) |
%\end{eqnarray*}
%
%Let's focus on the first term. 
%Decompose the first probability in three terms corresponding to the final choice of $(A',B')$, $(A',C')$ or $(B',C')$, and the second one in three terms according to the position of $\theta$ :
%\begin{eqnarray*}
%& & | \mathbb{P}(A'_1 , A'_2 \in U_0) - \mathbb{P}(A_1 , A_2 \in U_0) | \\
%&\leq& \frac{1}{3} | \mathbb{P}(A' \in U_0 , A' + e^{i \theta '} \in U_0) - \mathbb{P}(A_1 \in U_0 , A' + e^{i \theta} \in U_0 \ |\  \theta < \frac{\pi}{3} ) | \\
%&+& \frac{1}{3} | \mathbb{P}(A' \in U_0 , A' + e^{i(\theta ' + \pi / 3 ) } \in U_0) - \mathbb{P}(A_1 \in U_0 , A' + e^{i \theta} \in U_0 \ |\  \frac{\pi}{3} \leq \theta < \frac{2 \pi}{3} ) | \\
%&+& \frac{1}{3} | \mathbb{P}(A' + e^{i \theta '} \in U_0 , A' + e^{i(\theta ' + \pi / 3 ) } \in U_0) - \mathbb{P}(A_1 \in U_0 , A' + e^{i \theta} \in U_0 \ | \  \frac{2 \pi}{3} < \theta ) |
%\end{eqnarray*}
%
%Once again, let's focus on the first term. Let $\mathcal{P'}$ be the parallelogram  composed of the points of  $\mathcal{P}$ at a distance of at least one from the borders. Conditioning to the fact that the points $A$ and $A'$ fall on $\mathcal{P'}$, the two points are uniformly distributed in $\mathcal{P'}$, and independent of $\theta$ and $\theta'$. Thus this term is overestimated by the probability for $A$ and $A'$ to fall near the borders of the parallelogram. This probability is smaller than : $\mathcal{P}/\mathcal{A}$.
%
%\end{proof}
%
%
%\subsection*{Second proof}
%\fi


\end{document}
